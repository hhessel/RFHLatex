\section{HTTP}
Das Hypertext Transfer Protokoll, kurz HTTP, wird hauptsächlich benutzt um Webseiten über ein Netzwerk zu laden. Ein erster Entwurf des HTTP/1.0 Protokolls wurde 1996 von den Wissenschaftlern
Roy Fielding, Tim Berners-Lee und Anderen im CERN Institut als RFC 1945\footcite{BERNERS-LEE:RFC1945:Online} herausgegeben. Im Jahr 1999 folgte der RFC 2616\footcite{FIELDING:RFC2616:Online}, der die überarbeitete Version HTTP/1.1 beinhaltete.


Die Basis für die Kommunikation über HTTP ist die Notwendigkeit eines Servers und eines Client vorausgesetzt. Zwischen Server und Client werden
Nachrichten gesendet, die auf Clientseite Request heißen und auf Server Seite Response. Jede Nachricht besitzt ein Header und einen Body. Im Header
werden zusätzliche Informationen über die eigentlich Nachricht gespeichert, während der Body die eigentlichen Daten beinhaltet. Um einen Request zu erstellen muss eine von sieben
Request Methoden deklariert werden. 



In der HTTP/1.0 Protokoll Version werden Verbindung standardmäßig nach einer Response geschlossen. Im Gegensatz dazu kann man in der Version HTTP/1.1
ein s.g. Keep-Alive Flag setzen, welches es erlaubt die bestehende Verbindung nicht zu schließen und wieder zuverwenden. Dies erspart den Overhead
neue HTTP Nachrichten zu erstellen und spart entsprechend Ressourcen auf dem Server System. Mit Hilfe dieses optionalen Parameters wird die HTTP Verbindung
persistent. Das 1.1 Protokoll besitzt zudem die Möglichkeit mehrere Verbindungen zu bündeln (s.g. Pipelining). Mittels dieser Funktion
können z.B. Anfragen für unterschiedliche Ressourcen über dieselbe TCP/IP Verbindung übermittelt werden. Auch abgebrochene Verbindung können
in der neueren Version des Protokolls wieder aufgenommen werden. 

Die genannten Vorteile der Protokoll Version HTTP/1.1 spielen bei der späteren Simulation eine wichtige Rolle.

Ein typischer HTTP 1.1 Header stellt sich wie folgt dar:

\counterwithout{figure}{chapter} 
\begin{figure}[h]
\centering
\includegraphics[width=\textwidth]{images/abt1_http_header.jpg}
\caption{HTTP Request Header}
\end{figure}

Die bereits angesprochenen Flags, wie z.B. das Keep-Alive Flag werden gesetzt. Zusätzlich werden weitere Meta-Informationen über das Client System mitgeschickt. 

Die entsprechende Response auf das o.g. Request stellt sich wie folgt dar

\counterwithout{figure}{chapter} 
\begin{figure}[h]
\centering
\includegraphics[width=\textwidth]{images/abt2_http_header.jpg}
\caption{HTTP Response Header}
\end{figure}
