\documentclass[
12pt, %Schriftgröße
twoside=false, %Es wird nur einseitig gedruckt; Alternative: twoside
titlepage, %Es wird eine Titelseite ausgegeben
headings=small, %Größe der Kapitelüberschrift; Altern.: nichts/big (größer) bzw. small oder normal
listof=totoc, %Tabellen- und Abbildungsverzeichnis werden im Inhaltsverzeichnis ausgegeben
bibliography=totoc, %Literaturverzeichnis wird im Inhaltsverzeichnis ausgegeben
index=totoc, %Index wird im Inhaltsverzeichnis ausgegeben.
%chapterprefix, %Vor jedem Kapitel steht 'Kapitel' mit Absatz
appendixprefix, %Vor jedem Anhang steht 'Anhang' mit Absatz
%pdftex,
parskip,
final,
numbers=noendperiod
]{scrreprt}


\usepackage[%																							 Das Paket wird für bessere PDF-Ausgabe verwendet
	pdftitle={Strukturexposee Henrik P. Hessel BWI1102003},%             Titel des PDF Dokuments.
	pdfauthor={Henrik P. Hessel},%                            Autor des PDF Dokuments.
	pdfsubject={},%                 Thema des PDF Dokuments.
	pdfcreator={MiKTeX, LaTeX with hyperref and KOMA-Script},%Erzeuger des PDF Dokuments.
	pdfkeywords={}%      Schlüsselwörter für das PDF.
																								 %         (Diese werden von Suchmaschinen
		%                                                      auch für PDF Dokumente indexiert.)
	pdfpagemode=UseOutlines,%                                Inhaltsverzeichnis anzeigen beim Öffnen
	pdfdisplaydoctitle=true,%                                Dokumenttitel statt Dateiname anzeigen.
	pdflang=de,%  		
	urlbordercolor={1 1 1},
	linkbordercolor={1 1 1},
	citebordercolor={1 1 1},%  
	plainpages=false,
	hyperfootnotes=false,
	bookmarksopenlevel=1,
	bookmarksopen=true	                                    
]{hyperref}

\hypersetup{bookmarksnumbered=true}

\usepackage[paper=a4paper,left=3.5cm,right=2.5cm,top=3cm,bottom=2.0cm]{geometry}
\usepackage[ngerman]{babel} % Spracheinstellung: Deutsch
\usepackage[T1]{fontenc} % Für Sonderzeichen u.a.
\usepackage[utf8]{inputenc} % Direkte Eingabe der Umlaute

\usepackage{mathptmx}            % für Schrift mit Serifen
\usepackage[scaled=.90]{helvet}  % für Schrift ohne Serifen
\usepackage{courier}             % für Schrift mit konstanter Breite
\usepackage{chngcntr}
\usepackage{url}
\usepackage{tocloft}
\usepackage{acronym}
\usepackage[autostyle]{csquotes}
\usepackage{tabularx}
\usepackage[table]{xcolor}
\rowcolors{2}{gray!25}{white}
\usepackage{listings}
\usepackage{graphicx}

\definecolor{lightgrey}{rgb}{0.9,0.9,0.9}
\definecolor{darkgreen}{rgb}{0,0.6,0}

\lstset{language=[LaTeX]TeX,
texcsstyle=*\bf\color{blue},
numbers=none,
breaklines=true,
keywordstyle=\color{darkgreen},
commentstyle=\color{red},
otherkeywords={$, \{, \}, \[, \]},
frame=none,
tabsize=2,
backgroundcolor=\color{lightgrey},
basicstyle=\ttfamily\scriptsize,
showstringspaces=false}

\renewcommand{\lstlistlistingname}{Verzeichnis der Quellcodes}
\renewcommand{\lstlistingname}{Quellcode}

\setlength{\tabcolsep}{4pt}
\renewcommand{\arraystretch}{2}

\usepackage{scrpage2} 
\cfoot[]{} 
\ofoot[\pagemark]{\pagemark} 


\usepackage[
    backend=biber,
    style=authoryear-icomp,
    sortlocale=de_DE,
    natbib=true,
    url=false, 
    doi=true,
    eprint=false
]{biblatex}


%  TOC


%
%			Zeilenhöhe einstellen
%
\usepackage{setspace}

\setstretch{1,5} %Nur einschalten, wenn 1,5 Abstand von Word verlangt wird !!! Wird in der titlepage auf 1,5 gesetzt !!!
\onehalfspacing
\setlength{\headheight}{1.1\baselineskip}
%
%			Absatzhöhe einstellen
%
\parindent 0pt %Kein Einzug am Zeilenanfang von neuem Absatz
% \setlength\parskip{2ex} %Wenn \\ eingegeben wird, wird Abstand von 2x erzeugt
\renewcommand{\\}{\par} %Anstatt \par geht auch \\

\counterwithout{footnote}{chapter}

\makeatletter 
\g@addto@macro\UrlBreaks{ 
  \do\a\do\b\do\c\do\d\do\e\do\f\do\g\do\h\do\i\do\j 
  \do\k\do\l\do\m\do\n\do\o\do\p\do\q\do\r\do\s\do\t 
  \do\u\do\v\do\w\do\x\do\y\do\z\do\&\do\1\do\2\do\3 
  \do\4\do\5\do\6\do\7\do\8\do\9\do\0} 
% \def\do@url@hyp{\do\-} 
\makeatother 


\tolerance=2000 %Damit die Zeilen nicht überstehen -> emergencystretch
\setlength{\emergencystretch}{20pt}

\makeindex